%% LyX 2.2.3 created this file.  For more info, see http://www.lyx.org/.
%% Do not edit unless you really know what you are doing.
\documentclass{article}
\usepackage[latin9]{inputenc}
\usepackage{listings}
\renewcommand{\lstlistingname}{Listing}

\begin{document}

\part{Introduction}

CSCN files are easily editable text files (since the syntax is understood).

\part{Data types}

The data types used in scene files are these ones :

\emph{boolean} : a boolean (\emph{true} or \emph{false}).

\emph{int} : a simple integer.

\emph{real} : a floating point number, using dot ( . ) as the decimals
separator.

\emph{2, 3, 4 ints} : 2, 3 or 4 integers, separated by commas ( ,
) or spaces ( ).

\emph{2, 3, 4 reals} : 2, 3 or 4 floating point numbers, separated
by commas ( , ) or spaces ( ).

\emph{size} : 2 integers greater than or equal to 0.

\emph{2x2, 3x3, 4x4 reals matrix} : 2, 3 or 4 groups separated by
semi colons ( ; ) of 2, 3 or 4 floating point numbers separated by
commas ( , ) or spaces ( ).

\emph{rgb\_colour} : the RGB components of colour, expressed in floating
point numbers between 0.0 and 1.0.

\emph{rgba\_colour} : the RGBA components of colour, expressed in
floating point numbers between 0.0 and 1.0.

\emph{rgb\_hdr\_colour} : the RGB components of colour, expressed
in floating point numbers greater than or equal to 0.0.

\emph{rgba\_hdr\_colour} : the RGBA components of colour, expressed
in floating point numbers greater than or equal to 0.0.

\emph{value} : a string corresponding to a predefined value.

\emph{name} : a string, wrapped into double quotes ( \char`\"{} ).

\emph{file} : a string, wrapped into double quotes ( \char`\"{} ),
containing a file name and path.

\emph{folder} : a string, wrapped into double quotes ( \char`\"{}
), containing a folder path.

\part{Sections}

\section{Description}

The file is split into sections, defined as follows :

\begin{lstlisting}[section_type]
 "[section_name]"
{
	// Section description
}
\end{lstlisting}

Example:

\begin{lstlisting}
light "Light0"
{
	type directional
	colour 1.0 1.0 1.0
	intensity 0.8 1.0
}
\end{lstlisting}

Some sections can have child subsections :

\begin{lstlisting}
material "Bronze"
{
	pass
	{
		ambient 0.2125 0.1275 0.054
		diffuse 0.714 0.4284 0.12144
		emissive 0.0
		specular 0.393548 0.271906 0.166721
		shininess 25.6
	}
}
\end{lstlisting}

\subsection{Sections list}

The possible sections are the following:
\begin{enumerate}
\item 'sampler'

Defines a texture sampler object. 
\item 'material'

Defines a material. 
\item 'mesh'

Defines a mesh. 
\item 'font'

Defines a font used in text overlays. 
\item 'window'

Defines a render window. 
\item 'panel\_overlay'

Defines a simple panel overlay. 
\item 'border\_panel\_overlay'

Defines a panel overlay with a border. 
\item 'text\_overlay'

Defines a panel overlay with a text. 
\item 'scene'

Defines a whole scene. 
\end{enumerate}

\section{'sampler' section}
\begin{enumerate}
\item 'min\_filter' : \emph{value}

Value used for minification function. The possible values are : 
\begin{itemize}
\item \emph{linear} : linear interpolation. 
\item \emph{nearest} : no interpolation. 
\end{itemize}
\item 'mag\_filter' : \emph{value}

Value used for magnification function. The possible values are : 
\begin{itemize}
\item \emph{linear} : linear interpolation. 
\item \emph{nearest} : no interpolation. 
\end{itemize}
\item 'mip\_filter' : \emph{value}

Value used for mipmapping function. The possible values are : 
\begin{itemize}
\item \emph{linear} : linear interpolation. 
\item \emph{nearest} : no interpolation. 
\end{itemize}
\item 'min\_lod' : \emph{real}

Defines minimum level of detail value. 
\item 'max\_lod' : \emph{real}

Defines maximum level of detail value. 
\item 'lod\_bias' : \emph{real}

Defines the MIP-Level. 
\item 'u\_wrap\_mode' : \emph{value}

Defines the wrapping mode of the texture, for the U component. The
possible values are : 
\begin{itemize}
\item \emph{repeat} : The texture is repeated. 
\item \emph{mirrored\_repeat} : The texture is repeated, each instance of
2 being the mirror of the other one. 
\item \emph{clamp\_to\_border} : The texture is stretched, the object edge
colour is defined as the texture edge colour. 
\item \emph{clamp\_to\_edge} : The texture is stretched, the object edge
colour is defined as the average of the texture edge colour and the
border colour. 
\end{itemize}
\item 'v\_wrap\_mode' : \emph{value}

Defines the wrapping mode of the texture, for the V component. The
possible values are : 
\begin{itemize}
\item \emph{repeat} : The texture is repeated. 
\item \emph{mirrored\_repeat} : The texture is repeated, each instance of
2 being the mirror of the other one. 
\item \emph{clamp\_to\_border} : The texture is stretched, the object edge
colour is defined as the texture edge colour. 
\item \emph{clamp\_to\_edge} : The texture is stretched, the object edge
colour is defined as the average of the texture edge colour and the
border colour. 
\end{itemize}
\item 'w\_wrap\_mode' : \emph{value}

Defines the wrapping mode of the texture, for the W component. The
possible values are : 
\begin{itemize}
\item \emph{repeat} : The texture is repeated. 
\item \emph{mirrored\_repeat} : The texture is repeated, each instance of
2 being the mirror of the other one. 
\item \emph{clamp\_to\_border} : The texture is stretched, the object edge
colour is defined as the texture edge colour. 
\item \emph{clamp\_to\_edge} : The texture is stretched, the object edge
colour is defined as the average of the texture edge colour and the
border colour. 
\end{itemize}
\item 'border\_colour' : \emph{rgba\_colour}

Defines the non textured border colour. 
\item 'max\_anisotropy' : \emph{real}

Defines the maximum degree of anisotropy. 
\end{enumerate}

\section{'material' section}

Materials can be multi-pass, so you can declare more than one pass
subsection. 
\begin{enumerate}
\item 'pass' : \emph{new section}

Defines a new section describing a texture. 
\end{enumerate}

\subsection{'pass' section}
\begin{enumerate}
\item 'diffuse' : \emph{rgb\_colour}

Defines diffuse colour of the pass (legacy materials only). 
\item 'albedo' : \emph{rgb\_colour}

Defines albedo colour of the pass (not available on legacy materials). 
\item 'specular' : \emph{rgb\_colour}

Defines specular colour of the pass (not available with metallic/roughness
materials). 
\item 'metallic' : \emph{real (between 0 and 1)}

Defines the metallness of the pass (metallic/roughness materials only). 
\item 'shininess' : \emph{real (between 0 and 128)}

Defines the specular exponent of the pass (legacy materials only). 
\item 'glossiness' : \emph{real (between 0 and 1)}

Defines the glossiness of the pass (specular/glossiness materials
only). 
\item 'roughness' : \emph{real (between 0 and 1)}

Defines the roughness of the pass (metallic/roughness materials only). 
\item 'ambient' : \emph{real (between 0 and 1)}

Defines ambient factor of the pass (legacy materials only). 
\item 'emissive' : \emph{real (between 0 and 1)}

Defines emissive factor of the pass. 
\item 'alpha' : \emph{real (between 0 and 1)}

Defines the global alpha value for the pass. 
\item 'two\_sided' : \emph{boolean}

Tells the pass is two sided (true) or not (false). 
\item 'blend\_func' : src : \emph{value}, dst : \emph{value}

The two functions (source and destination) used during alpha blending
: 
\begin{itemize}
\item \emph{zero} : The target (src or dst) won't be used during alpha blending. 
\item \emph{one} : The target (src or dst) will be the only one used. 
\item \emph{src\_colour} : The target colour will be the source colour (dst
only). 
\item \emph{inv\_src\_colour} : The target colour will be one minus the
source colour (dst only). 
\item \emph{dst\_colour} : The target colour will be the destination colour
(src only). 
\item \emph{inv\_dst\_colour} : The target colour will be one minus the
destination colour (src only). 
\item \emph{src\_alpha} : The target alpha will be the source alpha (dst
only). 
\item \emph{inv\_src\_alpha} : The target alpha will be one minus the source
alpha (dst only). 
\item \emph{dst\_alpha} : The target alpha will be the destination alpha
(src only). 
\item \emph{inv\_dst\_alpha} : The target alpha will be one minus the destination
alpha (src only). 
\item \emph{src\_alpha\_sat} : Sets source alpha to 1. 
\end{itemize}
\item 'texture\_unit' : \emph{new section}

Defines a new section describing a texture unit. 
\item 'alpha\_blend\_mode' : \emph{value}

Alpha blending mode name, can be one of: 
\begin{itemize}
\item \emph{none} : No alpha blending. 
\item \emph{additive} : Source and destination alpha values are added. 
\item \emph{multiplicative} : Source and destination alpha values are multiplied. 
\end{itemize}
\item 'colour\_blend\_mode' : \emph{value}

Colour blending mode name, can be one of: 
\begin{itemize}
\item \emph{none} : No colour blending. 
\item \emph{additive} : Source and destination colour values are added. 
\item \emph{multiplicative} : Source and destination colour values are multiplied. 
\end{itemize}
\item 'alpha\_func' : func : \emph{value} ref-val: \emph{real}

Defines the way alpha rejection is applied to the texture. The second
parameter is the reference value used in alpha rejection function.
The first parameter values can be : 
\begin{itemize}
\item \emph{always} : The sample colour is always applied. 
\item \emph{less} : The sample colour is applied if its alpha component
is less than the second parameter. 
\item \emph{less\_or\_equal} : The sample colour is applied if its alpha
component is less than or equal to the second parameter. 
\item \emph{equal} : The sample colour is applied if its alpha component
is equal to the second parameter. 
\item \emph{not\_equal} : The sample colour is applied if its alpha component
is different from the second parameter. 
\item \emph{greater\_or\_equal} : The sample colour is applied if its alpha
component is greater than or equal to the second parameter. 
\item \emph{greater} : The sample colour is applied if its alpha component
is greater than the second parameter. 
\item \emph{never} : The sample colour is never applied. 
\end{itemize}
\item 'refraction\_ratio' : \emph{real}

Defines the refraction ratio of the pass. Note that if there is no
refraction map, the refraction is still applied, using only the skybox. 
\item 'subsurface\_scattering' : \emph{new section}

Defines a new section describing subsurface scattering for the pass. 
\item 'parallax\_occlusion' : \emph{boolean}

Enables or disables parallax occlusion mapping (needs a normal map
and a height map). 
\end{enumerate}

\subsubsection{'texture\_unit' section}
\begin{enumerate}
\item 'image' : \emph{file}

Defines the image file name. 
\item 'render\_target' : \emph{new section}

Defines a new section describing a render target. 
\item 'colour' : \emph{colour}

Defines the base blending colour. 
\item 'map\_type' : \emph{value}

Defines the way the texture is mapped to the object : 
\begin{itemize}
\item \emph{none} : No effect. 
\item \emph{reflection} : reflection mapping. 
\item \emph{sphere} : sphere mapping. 
\end{itemize}
\item 'rgb\_blend' : func : \emph{value} Arg0 : \emph{value} Arg1 : \emph{value}

Defines the texture behaviour during colour blending. The first parameter
is the blending function, the two other ones are the operands (Arg0
and Arg1) of this function. The firs parameter can take one of the
following values : 
\begin{itemize}
\item \emph{none} : None of the two operands is used. 
\item \emph{first\_arg} : Returns Arg0. 
\item \emph{add} : Returns Arg0 + Arg1. 
\item \emph{add\_signed} : Returns Arg0 + Arg1 - 0.5. 
\item \emph{modulate} : Returns Arg0 x Arg1. 
\item \emph{subtract} : Returns Arg0 - Arg1. 
\item \emph{dot3\_rgb} : Returns 4 x {[}((Arg0r - 0.5) x (Arg1r - 0.5))
+ ((Arg0g - 0.5) x (Arg1g - 0.5)) + ((Arg0b - 0.5) x (Arg1b - 0.5)){]}. 
\item \emph{dot3\_rgba} : Returns 4 x {[}((Arg0r - 0.5) x (Arg1r - 0.5))
+ ((Arg0g - 0.5) x (Arg1g - 0.5)) + ((Arg0b - 0.5) x (Arg1b - 0.5)){]}. 
\end{itemize}
The two operands can be one of the following values : 
\begin{itemize}
\item \emph{texture} : The current texture colour 
\item \emph{texture0} : The first texture colour 
\item \emph{texture1} : The second texture colour 
\item \emph{texture2} : The third texture colour 
\item \emph{texture3} : The fourth texture colour 
\item \emph{constant} : 
\item \emph{diffuse} : 
\item \emph{previous} : 
\end{itemize}
\item 'alpha\_blend' : func : \emph{value} Arg0 : \emph{value} Arg1 : \emph{value}

Defines the texture behaviour during alpha blending. The first parameter
is the blending function, the two other ones are the operands (Arg0
and Arg1) of this function. The first parameter can take one of the
following values : 
\begin{itemize}
\item \emph{none} : None of the two operands is used. 
\item \emph{first\_arg} : Returns Arg0. 
\item \emph{add} : Returns Arg0 + Arg1. 
\item \emph{add\_signed} : Returns Arg0 + Arg1 - 0.5. 
\item \emph{modulate} : Returns Arg0 x Arg1. 
\item \emph{subtract} : Returns Arg0 - Arg1. 
\end{itemize}
The two operands can be one of the following values : 
\begin{itemize}
\item \emph{texture} : The current texture colour 
\item \emph{texture0} : The first texture colour 
\item \emph{texture1} : The second texture colour 
\item \emph{texture2} : The third texture colour 
\item \emph{texture3} : The fourth texture colour 
\item \emph{constant} : 
\item \emph{diffuse} : 
\item \emph{previous} : 
\end{itemize}
\item 'channel' : \emph{value}

The channel at which the texture is bound. Can be one of the following
: 
\begin{itemize}
\item \emph{colour} : Base colour. 
\item \emph{ambient} : Ambient lighting colour. 
\item \emph{diffuse} : Diffuse lighting colour. 
\item \emph{normal} : Normals. 
\item \emph{specular} : Specular lighting colour. 
\item \emph{opacity} : Opacity. 
\item \emph{gloss} : Specular exponent. 
\end{itemize}
\item 'sampler' : \emph{name}

Defines the sampler object used by the texture. 
\end{enumerate}

\subsubsection{'shader\_program' section}
\begin{enumerate}
\item 'vertex\_program' : \emph{new section}

Defines a new section describing the vertex program. 
\item 'pixel\_program' : \emph{new section}

Defines a new section describing the pixel program. 
\item 'geometry\_program' : \emph{new section}

Defines a new section describing the geometry program. 
\item 'hull\_program' : \emph{new section}

Defines a new section describing the hull (tessellation control) program. 
\item 'domain\_program' : \emph{new section}

Defines a new section describing the domain (tessellation evaluation)
program. 
\item 'constants\_buffer' : \emph{new section}

Defines a new section dexcribing a constants buffer (uniform buffer). 
\end{enumerate}

\subsubsection{'vertex/pixel/geometry/hull/domain\_program' section}
\begin{enumerate}
\item 'file' : \emph{file}

Shader file name 
\item 'sampler' : \emph{name}

Creates a new variable of sample (1D, 2D, \ldots{}) type, for the
pixel shader. 
\item 'input\_type' : \emph{value}

Defines the input faces data type, for geometry shader. Can be one
of the following : 
\begin{itemize}
\item \emph{points} : Points. 
\item \emph{lines} : Disjoint lines. 
\item \emph{line\_loop} : Joint lines loop. 
\item \emph{line\_strip} : Joint lines. 
\item \emph{triangles} : Disjoint triangles. 
\item \emph{triangle\_strip} : Joint triangles. 
\item \emph{triangle\_fan} : Triangles joint using the first point. 
\item \emph{quads} : Disjoint quads. 
\item \emph{quad\_strip} : Joint quads. 
\item \emph{polygon} : Polygons. 
\end{itemize}
\item 'output\_type' : \emph{value}

Defines the geometry chader output data type. Can be one of the following
: 
\begin{itemize}
\item \emph{points} : Points. 
\item \emph{line\_strip} : Joint lines. 
\item \emph{triangle\_strip} : Joint triangles. 
\item \emph{quad\_strip} : Joint quads. 
\end{itemize}
\item 'output\_vtx\_count' : \emph{int}

Defines the geometry shader output vertices. 
\item 'variable' : \emph{new section}

Defines a new section describing a uniform variable. 
\end{enumerate}

\subsubsection{'constants\_buffer' section}
\begin{enumerate}
\item 'shaders' : \emph{bitwise ORed values}

Shader types to which this buffer applies, can be one of: 
\begin{itemize}
\item \emph{vertex} 
\item \emph{hull} 
\item \emph{domain} 
\item \emph{geometry} 
\item \emph{pixel} 
\item \emph{compute} 
\end{itemize}
\item 'variable' : \emph{name}, \emph{new section}

Defines a new section describing a variable for this buffer. 
\end{enumerate}

\subsubsection{'variable' section}
\begin{enumerate}
\item 'type' : \emph{value}

Variable type name, can be : 
\begin{itemize}
\item \emph{int} : 1 signed integer. 
\item \emph{uint} : 1 unsigned integer. 
\item \emph{float} : 1 simple precision floating point number. 
\item \emph{double} : 1 double precision floating point number. 
\item \emph{vec2i} : 2 signed integers. 
\item \emph{vec3i} : 3 signed integers. 
\item \emph{vec4i} : 4 signed integers. 
\item \emph{vec2f} : 2 simple precision floating point numbers. 
\item \emph{vec3f} : 3 simple precision floating point numbers. 
\item \emph{vec4f} : 4 simple precision floating point numbers. 
\item \emph{vec2d} : 2 double precision floating point numbers. 
\item \emph{vec3d} : 3 double precision floating point numbers. 
\item \emph{vec4d} : 4 double precision floating point numbers. 
\item \emph{mat2x2i} : 2x2 signed integers matrix. 
\item \emph{mat2x3i} : 2x3 signed integers matrix. 
\item \emph{mat2x4i} : 2x4 signed integers matrix. 
\item \emph{mat3x2i} : 3x2 signed integers matrix. 
\item \emph{mat3x3i} : 3x3 signed integers matrix. 
\item \emph{mat3x4i} : 3x4 signed integers matrix. 
\item \emph{mat4x2i} : 4x2 signed integers matrix. 
\item \emph{mat4x3i} : 4x3 signed integers matrix. 
\item \emph{mat4x4i} : 4x4 signed integers matrix. 
\item \emph{mat2x2f} : 2x2 simple precision floating point numbers matrix. 
\item \emph{mat2x3f} : 2x3 simple precision floating point numbers matrix. 
\item \emph{mat2x4f} : 2x4 simple precision floating point numbers matrix. 
\item \emph{mat3x2f} : 3x2 simple precision floating point numbers matrix. 
\item \emph{mat3x3f} : 3x3 simple precision floating point numbers matrix. 
\item \emph{mat3x4f} : 3x4 simple precision floating point numbers matrix. 
\item \emph{mat4x2f} : 4x2 simple precision floating point numbers matrix. 
\item \emph{mat4x3f} : 4x3 simple precision floating point numbers matrix. 
\item \emph{mat4x4f} : 4x4 simple precision floating point numbers matrix. 
\item \emph{mat2x2d} : 2x2 double precision floating point numbers matrix. 
\item \emph{mat2x3d} : 2x3 double precision floating point numbers matrix. 
\item \emph{mat2x4d} : 2x4 double precision floating point numbers matrix. 
\item \emph{mat3x2d} : 3x2 double precision floating point numbers matrix. 
\item \emph{mat3x3d} : 3x3 double precision floating point numbers matrix. 
\item \emph{mat3x4d} : 3x4 double precision floating point numbers matrix. 
\item \emph{mat4x2d} : 4x2 double precision floating point numbers matrix. 
\item \emph{mat4x3d} : 4x3 double precision floating point numbers matrix. 
\item \emph{mat4x4d} : 4x4 double precision floating point numbers matrix. 
\end{itemize}
\item 'count' : \emph{int}

Variable occurences count (array size). 
\item 'value' :

Variable value, depends on the chosen type. 
\end{enumerate}

\subsubsection{'subsurface\_scattering' section}
\begin{enumerate}
\item 'strength' : \emph{real}

Defines the strength of the effect. 
\item 'gaussian\_width' : \emph{real}

Defines the width of the Gaussian blur. 
\item 'transmittance\_profile' : \emph{new section}

Defines a new section describing the transmittance profile. 
\end{enumerate}

\subsubsection{'transmittance\_profile' section}
\begin{enumerate}
\item 'factor' : \emph{vec4f}

Defines the three RGB components of the colour, and the fourth component
is used for the exponent of that colour. 
\end{enumerate}

\section{'font' section}
\begin{enumerate}
\item 'file' : \emph{file}

Defines the file holding the font. 
\item 'height' : \emph{int}

Defines the height (precision) of the font. 
\end{enumerate}

\section{'scene' section}
\begin{enumerate}
\item 'ambient\_light' : \emph{colour}

Defines the ambient lighting colour. For PBR materials, defines the
influence of the IBL on the scene. 
\item 'background\_colour' : \emph{colour}

Defines the background colour. 
\item 'background\_image' : \emph{file}

Defines the background image. 
\item 'import' : \emph{file}

Allows scene import from a CSCN file or another file format supported
by Castor3D importer plug-ins. 
\item 'scene\_node' : \emph{new section}

Defines a new section describing a scene node for objects, lights
or billboards. 
\item 'camera\_node' : \emph{new section}

Defines a new section describing a scene node for cameras. 
\item 'light' : \emph{new section}

Defines a new section describing a light source. 
\item 'object' : \emph{new section}

Defines a new section describing an object. 
\item 'billboard' : \emph{new section}

Defines a new section describing a billboards list. 
\item 'camera' : \emph{new section}

Defines a new section describing a camera. 
\item 'panel\_overlay' : \emph{new section}

Defines a new section describing a simple panel overlay. 
\item 'border\_panel\_overlay' : \emph{new section}

Defines a new section describing a simple panel overlay with a border. 
\item 'text\_overlay' : \emph{new section}

Defines a new section describing a simple panel overlay with text. 
\item 'animated\_object\_group' : \emph{new section}

Defines a new section describing an animated object group, with common
animations. 
\item 'mesh' : \emph{new section}

Defines a new section describing a mesh, that can be used with one
or more objects. 
\item 'particle\_system' : \emph{new section}

Defines a new section describing a particle system. 
\item 'skybox' : \emph{new section}

Defines a new section describing the skybox. 
\item 'include' : \emph{file}

Includes a scene file, allowing you to split your scene in multiple
files. 
\item 'sampler' : \emph{new section}

Defines a new section describing a sampler. 
\item 'fog\_type' : \emph{value}

Defines the fog type for the scene. Possible values are: 
\begin{itemize}
\item \emph{linear}: Fog intensity increases linearly with distance to camera. 
\item \emph{exponential}: Fog intensity increases exponentially with distance
to camera. 
\item \emph{squared\_exponential}: Fog intensity increases even more with
distance to camera. 
\end{itemize}
\item 'fog\_density' : \emph{real}

Defines the fog density, which is multiplied by the distance, according
to chosen fog type. 
\item 'hdr\_config' : \emph{real}

Defines a new section describing the HDR configuration. 
\end{enumerate}

\subsection{'hdr\_config' section}
\begin{enumerate}
\item 'exposure' : \emph{real}

Defines the scene's exposure. 
\item 'gamma' : \emph{real}

Defines the gamma correction. 
\end{enumerate}

\subsection{'scene\_node' and 'camera\_node' sections}
\begin{enumerate}
\item 'parent' : \emph{name}

Defines this node's parent. The default parent node is the root node. 
\item 'position' : \emph{3 reals}

Node position relative to its parent. 
\item 'orientation' : \emph{4 reals}

A quaternion holding node orientation relative to its parent. 
\item 'scale' : \emph{3 reals}

Node scale relative to its parent. 
\end{enumerate}

\subsection{'light' section}
\begin{enumerate}
\item 'type' : \emph{value}

Three light source types exist in Castor3D : 
\begin{itemize}
\item \emph{directional} : directional light (like the sun). 
\item \emph{point\_light} : a positioned source which emits in all directions. 
\item \emph{spot\_light} : a positioned source which emits in an oriented
cone. 
\end{itemize}
\item 'colour' : \emph{rgb\_colour}

Defines the colour for this source. 
\item 'intensity' : \emph{2 reals}

Defines the diffuse and specular intensities for this source. 
\item 'attenuation' : \emph{3 reals}

Defines the three attenuation components : constant, linear and quadratic.
This attenuation is computed from the distance to the light source.

Only for spot\_light and point\_light. 
\item 'cut\_off' : \emph{real}

Defines the angle of the emission cone.

Only for spot\_light. 
\item 'exponent' : \emph{real}

Defines the attenuation computed with the distance from the emission
cone centre.

Only for spot\_light. 
\item 'parent' : \emph{name}

Defines the node which this light source is attached to. 
\item 'shadow\_producer' : \emph{boolean}

Defines if the light produces shadows (\emph{true}) or not (\emph{false},
default value). 
\end{enumerate}

\subsection{'object' section}
\begin{enumerate}
\item 'parent' : \emph{name}

Defines the node which this object is attached to. 
\item 'mesh' : \emph{name}

Defines the mesh this object uses. 
\item 'mesh' : \emph{name} \emph{new section}

Defines a new section describing a mesh with the given name. 
\item 'material' : \emph{name}

Name of a material, defined in a .cmtl file or in this file. Applies
this material too all the submeshes. 
\item 'materials' : \emph{new section}

New section used to specify each submesh's material. 
\item 'cast\_shadows' : \emph{boolean}

Defines if the object casts shadows (\emph{true}, default value) or
not (\emph{false}). 
\item 'receive\_shadows' : \emph{boolean}

Defines if the object receives shadows (\emph{true}, default value)
or not (\emph{false}). 
\end{enumerate}

\subsubsection{'materials' section}
\begin{enumerate}
\item 'material' : \emph{int}, \emph{name}

Submesh index, and material name for this submesh. 
\end{enumerate}

\subsection{'billboard' section}

Allows the definition of billboards that share the same material and
dimensions. 
\begin{enumerate}
\item 'parent' : \emph{name}

Defines the parent scene node. 
\item 'positions' : \emph{new section}

Defines a new section describing each billboard position. 
\item 'material' : \emph{name}

Defines the material used by every billboard of this list. 
\item 'dimensions' : \emph{size}

Defines billboards dimensions. 
\item 'type' : \emph{value}

Defines the type of billboard. Possible values are: 
\begin{itemize}
\item \emph{cylindrical}: The billboard faces the camera, except for their
Y axis, which remains still. 
\item \emph{spherical}: The billboard faces the camera on all axes. 
\end{itemize}
\item 'size' : \emph{value}

Defines the billboards sizing. Possible values are: 
\begin{itemize}
\item \emph{dynamic}: The size varies, depending on the distance to camera. 
\item \emph{fixed}: The size is fixed, where the camera is. 
\end{itemize}
\end{enumerate}

\subsubsection{'positions' section}
\begin{enumerate}
\item 'pos' : \emph{3 reals} 

Defines the relative position of a billboard. 
\end{enumerate}

\subsection{'camera' section}
\begin{enumerate}
\item 'parent' : \emph{name}

Defines the parent CameraNode. 
\item 'primitive' : \emph{value}

Defines the display type. Can be one of : 
\begin{itemize}
\item \emph{points} : Points. 
\item \emph{lines} : Disjointed lines. 
\item \emph{line\_loop} : Jointed lines loop. 
\item \emph{line\_strip} : Joined lines. 
\item \emph{triangles} : Disjointed triangles. 
\item \emph{triangle\_strip} : Jointed triangles. 
\item \emph{triangle\_fan} : Triangles jointed using the first point. 
\item \emph{quads} : Disjointed quads. 
\item \emph{quad\_strip} : Jointed quads. 
\item \emph{polygon} : Polygons. 
\end{itemize}
\item 'viewport' : \emph{new section}

Defines the camera view port. 
\end{enumerate}

\subsubsection{'viewport' section}
\begin{enumerate}
\item 'type' : \emph{value}

Window display type, 2d or 3d. 
\item 'left' : \emph{real}

Defines the minimum displayed X coordinate. 
\item 'right' : \emph{real}

Defines the maximum displayed X coordinate. 
\item 'top' : \emph{real}

Defines the minimum displayed Y coordinate. 
\item 'bottom' : \emph{real}

Defines the maximum displayed Y coordinate. 
\item 'near' : \emph{real}

Defines the minimum displayed Z coordinate. 
\item 'far' : \emph{real}

Defines the maximum displayed Z coordinate. 
\item 'size' : \emph{size}

Defines the window display size (in pixels). 
\item 'fov\_y' : \emph{real}

Defines the vertical field of view angle, in radians. 
\item 'aspect\_ratio' : \emph{real}

Defines the global window aspect ratio (1.33333 for 4/3, 1.77777 for
16/9 \ldots{} ). 
\end{enumerate}

\subsection{'animated\_object\_group' section}
\begin{enumerate}
\item 'animated\_object' : \emph{name}

Adds the object with the given name to the group. 
\item 'animation' : \emph{name}

Adds the animation with the given name to the group's common animations
list. 
\item 'start\_animation' : \emph{name}

Starts the given animation. 
\item 'pause\_animation' : \emph{name}

Pauses the given animation (which must have been started first). 
\end{enumerate}

\subsubsection{'animation' section}
\begin{enumerate}
\item 'looped' : \emph{boolean}

Defines if the animation is looped (\emph{true}) or not (\emph{false},
default value). 
\item 'scale' : \emph{real}

Defines the time scale of the animation (can be negative, the animation
will then be played backwards). 
\end{enumerate}

\section{'mesh' section}
\begin{enumerate}
\item 'type' : \emph{name}

Mesh type name, one of : 
\begin{itemize}
\item \emph{custom} : manually defined mesh or imported mesh. 
\item \emph{cube} : a cube, user must then define its three dimensions. 
\item \emph{cone} : a cone, user must then define its radius and height. 
\item \emph{cylinder} : a cylinder, user must then define its radius and
height. 
\item \emph{sphere} : a sphere with quad faces, user must then define the
subdivisions count and the radius. 
\item \emph{icosahedron} : a sphere with triangular faces, user must then
define the subdivisions count and the radius. 
\item \emph{torus} : a torus, user must then define the internal and external
subdivisions count and the internal and external radius. 
\item \emph{plane} : a plane, user must then define the width and depth
subdivisions count and the width and depth. 
\end{itemize}
\item 'submesh' : \emph{new section}

Defines a new section describing a submesh. Only if the mesh type
is custom. 
\item 'import' : \emph{file} \emph{\textless{}options\textgreater{}}

Allows import of mesh data from a file, in CMSH file format or any
format supported by Castor3D import plug-ins. Only if the mesh type
is custom. This directive can accept few optional parameters : 
\begin{itemize}
\item \emph{smooth\_normals} : Computes normals per vertex during import. 
\item \emph{flat\_normals} : Computes normals per face during the import. 
\item \emph{tangent\_space} : Computes tangent space informations (tangent
and bi-tangent) during import. 
\item \emph{rescale}=\emph{real} : Rescales the resulting mesh by given
factor, on three axes. 
\end{itemize}
\item 'morph\_import' : \emph{file} \emph{\textless{}options}\textgreater{}

Allows import of mesh data from a file, to add mophing animation.
This directive must happen after a first import directive. Only if
the mesh type is custom. This directive can accept few optional parameters
: 
\begin{itemize}
\item \emph{smooth\_normals} : Computes normals per vertex during import. 
\item \emph{flat\_normals} : Computes normals per face during the import. 
\item \emph{tangent\_space} : Computes tangent space informations (tangent
and bi-tangent) during import. 
\item \emph{rescale}=\emph{real} : Rescales the resulting mesh by given
factor, on three axes. 
\end{itemize}
\item 'division' : \emph{name} \emph{int}

Allows the mesh subdivision, using a supported Castor3D divider plug-in
algorithm. The second parameter is the application count of the algorithm
(its applied recursively). 
\end{enumerate}

\subsection{'submesh' section}
\begin{enumerate}
\item 'vertex' : \emph{3 reals}

Defines a vertex position. 
\item 'uv' : \emph{2 reals}

Defines the UV texture coordinates for the previously declared vertex. 
\item 'uvw' : \emph{3 reals}

Defines the UVW texture coordinates for the previously declared vertex. 
\item 'normal' : \emph{3 reals}

Defines the normal coordinates for the previously declared vertex. 
\item 'tangent' : \emph{3 reals}

Defines the tangent coordinates for the previously declared vertex. 
\item 'face' : \emph{3 or 4 integers}

Defines a face using the three or four vertices whose indices are
given. If more than three indices are given, creates the appropriate
count of triangular faces. 
\item 'face\_uv' : \emph{as much uv as the face indices}

Defines the UV coordinates for each vertex of the previously declared
face. 
\item 'face\_uvw' : \emph{as much uvw as the face indices}

Defines the UVW coordinates for each vertex of the previously declared
face. 
\item 'face\_normals' : \emph{as much 3 reals groups as the face indices}

Defines the normals coordinates for each vertex of the previously
declared face. 
\item 'face\_tangents' : \emph{as much 3 reals groups as the face indices}

Defines the tangents coordinates for each vertex of the previously
declared face. 
\end{enumerate}

\section{'panel\_overlay' section}
\begin{enumerate}
\item 'material' : \emph{name}

Defines the material used by the panel. 
\item 'position' : \emph{2 reals}

Defines the overlay position, relative to its parent (or to screen,
if no parent). 
\item 'size' : \emph{2 reals}

Defines the overlay size, relative to its parent (or to screen, if
no parent). 
\item 'pxl\_position' : \emph{2 ints}

Defines the absolute position for the overlay, in pixels. 
\item 'pxl\_size' : \emph{2 ints}

Defines the absolute size for the overlay, in pixels. 
\item 'uv' : \emph{4 reals}

Defines the UV for the overlay (left, top, right, bottom). 
\item 'panel\_overlay' : \emph{name} \emph{new section}

Defines a new section describing a simple panel children overlay. 
\item 'border\_panel\_overlay' : \emph{name} \emph{new section}

Defines a new section describing a border panel children overlay. 
\item 'text\_overlay' : \emph{name} \emph{new section}

Defines a new section describing a text panel children overlay. 
\end{enumerate}

\section{'border\_panel\_overlay' section}
\begin{enumerate}
\item 'material' : \emph{name}

Defines the material used by the panel. 
\item 'position' : \emph{2 reals}

Defines the overlay position, relative to its parent (or to screen,
if no parent). 
\item 'size' : \emph{2 reals}

Defines the overlay size, relative to its parent (or to screen, if
no parent). 
\item 'pxl\_position' : \emph{2 ints}

Defines the absolute position for the overlay, in pixels. 
\item 'pxl\_size' : \emph{2 ints}

Defines the absolute size for the overlay, in pixels. 
\item 'center\_uv' : \emph{4 reals}

Defines the UV for the center of the overlay (left, top, right, bottom). 
\item 'border\_material' : \emph{name}

Defines the material used for the panel's border. 
\item 'border\_position' : \emph{value}

Defines the border's position, can be one of: 
\begin{itemize}
\item \emph{internal} : The border is inside the overlay. 
\item \emph{middle} : The border is half inside, half outside the overlay. 
\item \emph{external} : The border is outside the overlay. 
\end{itemize}
\item 'border\_size' : \emph{4 reals}

Defines the borders sizes (left right, top, bottom). 
\item 'pxl\_border\_size' : \emph{2 ints}

Defines the absolute border size, in pixels. 
\item 'border\_inner\_uv' : \emph{4 reals}

Defines the UV for the border of the overlay, inner side (left, top,
right, bottom). 
\item 'border\_outer\_uv' : \emph{4 reals}

Defines the UV for the border of the overlay, outer side (left, top,
right, bottom). 
\item 'panel\_overlay' : \emph{name} \emph{new section}

Defines a new section describing a simple panel children overlay. 
\item 'border\_panel\_overlay' : \emph{name} \emph{new section}

Defines a new section describing a border panel children overlay. 
\item 'text\_overlay' : \emph{name} \emph{new section}

Defines a new section describing a text panel children overlay. 
\end{enumerate}

\section{'text\_overlay' section}
\begin{enumerate}
\item 'material' : \emph{name}

Defines the material used by the panel. 
\item 'position' : \emph{2 reals}

Defines the overlay position, relative to its parent (or to screen,
if no parent). 
\item 'size' : \emph{2 reals}

Defines the overlay size, relative to its parent (or to screen, if
no parent). 
\item 'pxl\_position' : \emph{2 ints}

Defines the absolute position for the overlay, in pixels. 
\item 'pxl\_size' : \emph{2 ints}

Defines the absolute size for the overlay, in pixels. 
\item 'font' : \emph{name}

Defines the font used to display the text. 
\item 'text' : \emph{texte}

Defines the displayed text. 
\item 'text\_wrapping' : \emph{value}

Defines the way the text is cut when a line overflows the overlay
dimensions. Can be one of: 
\begin{itemize}
\item \emph{none} : The text is not cut (the part that overflows won't be
displayed, though). 
\item \emph{break} : The text is cut at the letter (the words will be cut). 
\item \emph{break\_words} : The text is cut at the word (the words remain
uncut). 
\end{itemize}
\item 'vertical\_align' : \emph{value}

Defines the text vertical alignment: 
\begin{itemize}
\item \emph{top} : Align on top. 
\item \emph{center} : Vertically center. 
\item \emph{bottom} : Align on bottom. 
\end{itemize}
\item 'horizontal\_align' : \emph{value}

Defines the text horizontal alignment: 
\begin{itemize}
\item \emph{left} : Align on left. 
\item \emph{center} : Horizontally center. 
\item \emph{right} : Align on right. 
\end{itemize}
\item 'texturing\_mode' : \emph{value}

Defines the way the texture is applied: 
\begin{itemize}
\item \emph{letter} : The texture is applied on each letter. 
\item \emph{text} : The texture is applied on the whole text. 
\end{itemize}
\item 'line\_spacing\_mode' : \emph{value}

Defines the height of the lines: 
\begin{itemize}
\item \emph{own\_height} : Each line has its own height. 
\item \emph{max\_lines\_height} : Each line has the height of the line with
the maximum height. 
\item \emph{max\_font\_height} : Each line has the height of the character
with the maximum height in the font. 
\end{itemize}
\item 'panel\_overlay' : \emph{name} \emph{new section}

Defines a new section describing a simple panel children overlay. 
\item 'border\_panel\_overlay' : \emph{name} \emph{new section}

Defines a new section describing a border panel children overlay. 
\item 'text\_overlay' : \emph{name} \emph{new section}

Defines a new section describing a text panel children overlay. 
\end{enumerate}

\section{Section 'window'}
\begin{enumerate}
\item 'render\_target' : \emph{new section}

Defines a new section describing the render target. 
\item 'vsync' : \emph{boolean}

Defines the activation or deactivation of vertical synchronisation. 
\item 'fullscreen' : \emph{boolean}

Defines the activation or deactivation of full-screen display. 
\end{enumerate}

\section{Section 'render\_target'}
\begin{enumerate}
\item 'scene' : \emph{nom}

Defines the scene rendered in this target. 
\item 'camera' : \emph{nom}

Defines the camera used to render the scene. 
\item 'size' : \emph{size}

Defines the internal buffer dimensions. 
\item 'format' : \emph{value}

Defines the colour buffer pixel format. Can be one of : 
\begin{itemize}
\item \emph{l8} : Luminance 8 bits, one 8 bits integral number. 
\item \emph{l16f} : Luminance 16 bits, one 16 bits floating point number
(half float). 
\item \emph{l32f} : Luminance 32 bits, one 32 bits floating point number(float). 
\item \emph{al16} : Alpha + Luminance, two 8 bits integral number. 
\item \emph{al16f} : Alpha + Luminance, two 16 bits floating point number
(half float). 
\item \emph{al32f} : Alpha + Luminance, two 32 bits floating point number
(float). 
\item \emph{argb1555} : ARGB 16 bits, 1 bit for alpha and each other component
on a 5 bits integer. 
\item \emph{rgb565} : RGB 16 bits, R and B on a 5 bits integer, G on a 6
bits integer. 
\item \emph{argb16} : ARGB 16 bits, each component on a 4 bits integer. 
\item \emph{rgb24} : RGB 24 bits, each component on a 8 bits integer. 
\item \emph{argb32} : ARGB 32 bits, each component on a 8 bits integer. 
\item \emph{argb16f} : ARGB 64 bits, each component on a 16 bits floating
point number (half float). 
\item \emph{rgb32f} : RGB 96 bits, each component on a 32 bits floating
point number (half float). 
\item \emph{argb32f} : ARGB 128 bits, each component on a 32 bits floating
point number (half float). 
\end{itemize}
\item 'depth' : \emph{value}

Defines the depth/stencil buffer pixel format. Can be one of : 
\begin{itemize}
\item \emph{depth16} : Depth on a 16 bits integer. 
\item \emph{depth24} : Depth on a 24 bits integer. 
\item \emph{depth24s8} : Depth on a 24 bits integer, Stencil on a 8 bits
integer. 
\item \emph{depth32fs8} : Depth on a 32 bits floating point, Stencil on
a 8 bits integer. 
\item \emph{depth32} : Depth on a 32 bits integer. 
\item \emph{depth32f} : Depth on a 32 bits floating point. 
\item \emph{stencil1} : Stencil on 1 bit. 
\item \emph{stencil8} : Stencil on a 8 bits integer. 
\end{itemize}
\item 'postfx' : \emph{value}

Defines a post render effect to use. The parameters depend on the
chosen effect. 
\item 'stereo' : \emph{boolean}

Tells if we use stereoscopic display mode. 
\item 'tone\_mapping' : \emph{name}

Defines the tone mapping operator to use with the render target. 
\item 'ssao' : \emph{new section}

Defines a new section describing the Screen Space Ambient Occlusion.
\end{enumerate}

\subsection{'ssao' section}
\begin{enumerate}
\item 'enabled' : \emph{boolean}

Defines the activation status of SSAO. 
\item 'high\_quality' : \emph{boolean}

Defines the quality of the effect. 
\item 'radius' : \emph{real}

Defines the radius of the effect (expressesd in meters). 
\item 'bias' : \emph{real}

Defines the bias of the effect. 
\item 'intensity' : \emph{real}

Defines the intensity of the effect. 
\item 'num\_samples' : \emph{int}

Defines the number of samples per pixel. 
\item 'edge\_sharpness' : \emph{real}

Defines the edge sharpness, in the blur pass. 
\item 'blur\_step\_size' : \emph{int}

Defines the size of a step in the blur pass. 
\item 'blur\_radius' : \emph{int}

Defines the blur radius. 
\end{enumerate}

\end{document}
